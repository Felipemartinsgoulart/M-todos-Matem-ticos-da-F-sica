\documentclass[a4paper]{article}
\usepackage[utf8]{inputenc}
\usepackage{amsmath}
\usepackage{graphicx}
\usepackage[bottom=2.0cm,top=2.0cm,left=2.0cm,right=2.0cm]{geometry}


\title{%
                    \begin{figure}[htp]
                        \centering
                        \includegraphics[width=5cm]{logo_unb.png}
                    \end{figure}
  Métodos Matemáticos da Física \\
  \large Departamento de Matemática - Universidade de Brasília}

\author{
    Felipe Martins Goulart - 15/0153651 \\
    Aluno  - 00/00000 \\
}
\date{Março 2021}

\begin{document}


\maketitle
 \renewcommand{\theenumi}{\alph{}}
\section {É Verdadeiro(V) ou Falso(F)?}


 \begin{enumerate}
   \begin{enumerate}
     \item Desenvolvendo a  função $ f(x) = x^2, - \pi  < x < \pi$, em série de Fourier com período de $2\pi$ e utilizando o Teorema de Fourier tem-se que
        \begin{equation}
            \begin{aligned}
                {\frac{1}{1^2}+\frac{1}{2^2}+...+\frac{1}{n^2}+...=\frac{\pi^2}{6}.}\\
            \end{aligned}
        \end{equation}
        
     \item Existe função $f \epsilon \mathit{L}^2([-\pi,\pi])$ de período $2\pi$ de tal modo que sua ´serie de Fourier seja $\sum_{n=1}^{- \infty} n^2sin(nx)$.
    
   \end{enumerate}
 \end{enumerate}


\section{Demonstre que \{$\sqrt{\frac{2}{L}} sin\frac{k\pi}{L}x \}_{k \epsilon N}$ é um \textbf{C.O.N.C.} em  $C_{per} ([0,L])$.}
    
\section{Demonstre que se $f \epsilon \mathit{L}^2([-L,L])$ então $\frac{a_0^2}{2} +\sum_{n=1}^{- \infty} \a_k^2 + b_k^2 = \frac{1}{L} \int_{-L}^{L} |f(x)|^2dx$, onde $a_0$, $a_k$, $b_{k}$ são coeficientes de Fourier.} 

\section{Encontre a série de Fourier da função f(x) = cos^5x }

\section{Explique porque a série $\sum_{n=1}^{- \infty} \frac{cos(nx)}{n}$ não é a série de Fourier de uma função diferneciável}



\end{document}
